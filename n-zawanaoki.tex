% このページレイアウトは http://hooktail.org/computer/index.php?TeX%A5%C6%A5%F3%A5%D7%A5%EC%A1%BC%A5%C8#content_1_3 より
\documentclass[11pt,a4paper,twocolumn]{jsarticle}

\usepackage{amsmath,amssymb}
\usepackage{bm}
\usepackage{graphicx}
\usepackage{ascmac}

\setlength{\textwidth}{\fullwidth}
\setlength{\textheight}{40\baselineskip}
\addtolength{\textheight}{\topskip}
\setlength{\voffset}{-0.2in}
\setlength{\topmargin}{0pt}
\setlength{\headheight}{0pt}
\setlength{\headsep}{0pt}

\begin{document}


{\bf 問い}: $n$沢直樹は, $n>0$の時は$1/n$倍返しをし, $n \le 0$では仕返ししない. この時, $n := m(m+3)(m-3)$沢直樹は, $n^2+m^2 \ge 16$の下では最大で何倍返し出来るか. [http://\allowbreak{}twitter.com/seirin0919/status/372358335053\allowbreak{}328384](句読点・数式の書き換えおよび自明な誤りの修正は引用者)

\bigskip

{\bf 回答}:

\smallskip
\noindent $n$沢直樹がする仕返しの倍率を$f(n)$とする.
すなわち
\[ f(n) := \begin{cases}
  1 / n & (n > 0) \\
  0 & (n \le 0)
\end{cases} \]
与えられた条件の下で$f(n)$の最大値を求めればよい.


\end{document}
