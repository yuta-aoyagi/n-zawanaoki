% このページレイアウトは http://hooktail.org/computer/index.php?TeX%A5%C6%A5%F3%A5%D7%A5%EC%A1%BC%A5%C8#content_1_3 より
\documentclass[11pt,a4paper,twocolumn]{jsarticle}

\usepackage{amsmath,amssymb}
\usepackage{bm}
\usepackage[dvipdfmx]{graphicx}
\usepackage{ascmac}

\setlength{\textwidth}{\fullwidth}
\setlength{\textheight}{40\baselineskip}
\addtolength{\textheight}{\topskip}
\setlength{\voffset}{-0.2in}
\setlength{\topmargin}{0pt}
\setlength{\headheight}{0pt}
\setlength{\headsep}{0pt}

\begin{document}


{\bf 問い}: $n$沢直樹は, $n>0$の時は$1/n$倍返しをし, $n \le 0$では仕返ししない. この時, $n := m(m+3)(m-3)$沢直樹は, $n^2+m^2 \ge 16$の下では最大で何倍返し出来るか. [http://\allowbreak{}twitter.com/seirin0919/status/372358335053\allowbreak{}328384](句読点・数式の書き換えおよび自明な誤りの修正は引用者)

\bigskip

{\bf 回答}:

\smallskip
\noindent $n$沢直樹がする仕返しの倍率を$f(n)$とする.
すなわち
\[ f(n) := \begin{cases}
  1 / n & (n > 0) \\
  0 & (n \le 0)
\end{cases} \]
与えられた条件の下で$f(n)$の最大値を求めればよい.

$m = -2$のとき, $n = -2 \cdot 1 \cdot -4 = 8$であるから, $n^2 + m^2 = 68 \ge 16$を満たす.
このとき$f(n) = 1/8 > 0$であるから, $f(n)$の最大値が$0$であることはない.
したがって$f(n)$の最大値を求めるには条件を満たす正の$n$の最小値を求めればよい.

$n^2 + m^2 \ge 16$は$mn$-平面で原点を中心とする半径$4$の円(以下$C_1$という)とその外側であり,
$n = m(m+3)(m-3)$は$m = \pm \sqrt 3$で極値$n = \mp 6 \sqrt 3$をとり変曲点$(m, n) = (0, 0)$を通る三次曲線(以下$C_2$という)である.
これを図示すると次のようになる.
\begin{figure}[h]
  \begin{center}
    \includegraphics[width=6.5cm]{graph.png}
  \end{center}
\end{figure}

図より$C_1$と$C_2$は$m$-軸より上では$-3 < m < - \sqrt 3$, $- \sqrt 3 < m < 0$そして$3 < m < 4$で交わる.
これらの交点を順に$P_1$, $P_2$そして$P_3$とする.

また図より, 求める正の$n$の最小値は上で見た交点の$n$-座標である.
交点を求めるため$C_1$と$C_2$の方程式を連立した
\[ \begin{cases}
  n^2 + m^2 = 16 \\
  n = m(m+3)(m-3)
\end{cases} \]
を$n > 0$の下で解く.
第2式を第1式に代入して整理すると
\begin{eqnarray*}
  && [m(m+3)(m-3)]^2 + m^2 = 16 \\
  & \Leftrightarrow & m^2[(m^2 - 9)^2 + 1] - 16 = 0 \\
  & \Leftrightarrow & m^6 - 18m^4 + 82m^2 - 16 = 0 \\
  & \Leftrightarrow & (m^2 - 8)(m^4 - 10m^2 + 2) = 0
\end{eqnarray*}
を得る.
したがって$m^2 = 8, \ 5 \pm \sqrt{23}$である.
$n^2 = 16 - m^2$であるから, 正で最小の$n$を与えるのは最大の$m^2$である$m^2 = 5 + \sqrt{23}$である.
よって$n^2 = 11 - \sqrt{23}$.

$C_2$の方程式と$n > 0$から$m$と$n$の符号が定まり, $P_3$の座標
\[ (m, n) = \biggl( \sqrt{5 + \sqrt{23}} \ , \sqrt{11 - \sqrt{23}} \ \biggr) \]
が分かる.

以上より, 求める最大の仕返しの倍率は
\[ f(n) = \frac{1}{\sqrt{11 - \sqrt{23}}} \approx 0.40 \]
である.


\end{document}
